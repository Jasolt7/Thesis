%!TEX root = ../template.tex
%%%%%%%%%%%%%%%%%%%%%%%%%%%%%%%%%%%%%%%%%%%%%%%%%%%%%%%%%%%%%%%%%%%%
%% abstract-pt.tex
%% NOVA thesis document file
%%
%% Abstract in Portuguese
%%%%%%%%%%%%%%%%%%%%%%%%%%%%%%%%%%%%%%%%%%%%%%%%%%%%%%%%%%%%%%%%%%%%

\typeout{NT FILE abstract-pt.tex}%

Os custos associados ao setor de saúde têm vindo a aumentar, consequentemente será necessário encontrar maneiras de atenuar este crescimento. A solução passará pela utilização eficiente dos recursos disponível. Uma das vertentes menos exploradas no setor de saúde para este efeito é o escalonamento eficiente de cada exame ao longo do dia de trabalho.\\
Desta forma, através da exploração do departamento de medicina nuclear no Hospital Garcia de Orta, foram desenvolvidos vários modelos com o objetivo de ajudar os decisores a responder a duas perguntas: Quanto tempo é necessário para executar todos os exames; Quantos exames podem ser feitos com o tempo disponível. O problema em estudo é uma generalização do problema \textit{Job-Shop}, em específico \textit{Flexible Multi-Resource Job-Shop with No-Wait}.\\
Esta dissertação é repartidas em três passos, no mapeamento dos exames e a respetiva recolha de dados, no formulação dos vários modelos, e na comparação dos modelos e resolução dos problemas práticos apresentados.\\
O mapeamento dos exames foi realizado através de entrevistas sucessivas com vários profissionais de forma a captar os recursos necessários em cada atividade, de seguida procedeu-se à recolha das duração de cada atividade.\\
Para cada uma das perguntas a responder foram criados quatro modelos, uma nova formulação \textit{Mixed Integer Linear Programming} e três versões de \textit{Simulated Annealing}, para esta última utilizou-se dois tipos de codificação para a solução.\\
Os modelos utilizados foram comparados utilizando problemas \textit{benchmark} já existentes na literatura e através de problemas práticos apresentados pelo sistema em estudo. Contudo, não foi possível aplicar na prática a agenda gerada pelos modelos.\\
Esta dissertação permitiu elaborar um conjuntos de modelos que aumentam a eficiência com a qual decorre o escalonamento de exames, não só permitindo automatizar este processo como também aumentando o volume de exames que ocorrem num dia de trabalho.

\keywords{
	Saúde \and
	Otimização \and
	Escalonamento \and
	\textit{Job-Shop} \and
	Mapeamento de processos
}