%!TEX root = ../template.tex
%%%%%%%%%%%%%%%%%%%%%%%%%%%%%%%%%%%%%%%%%%%%%%%%%%%%%%%%%%%%%%%%%%%%
%% Intro.tex
%% NOVA thesis document file
%%
%% Chapter with content
%%%%%%%%%%%%%%%%%%%%%%%%%%%%%%%%%%%%%%%%%%%%%%%%%%%%%%%%%%%%%%%%%%%%

\typeout{NT FILE Intro.tex}%

\chapter{Intrudução}
\label{cha:intrudução}

\prependtographicspath{{Chapters/Figures/Covers/}}

% epigraph configuration
\epigraphfontsize{\small\itshape}
\setlength\epigraphwidth{12.5cm}
\setlength\epigraphrule{0pt}

Este capítulo irá apresentar o contexto desta dissertação, bem como a motivação para a sua execução, os objetivos a atingir e a abordagem tomada.


\section{Contexto e motivação}
\label{sec:contexto_e_motivação}

Serviços de saúde representam uma carga considerável sobre o GDP mundial, sendo em 2021 de 10,36, é possível distinguir os países emergentes daqueles já desenvolvidos. Este segundo grupo gasta perto do dobro do seu GDP em relação ao primeiro, contudo a origem destes valores difere. Pouco mais de metade deste custo é suportado pelo governo, em países emergentes, e perto de 3/4 em países desenvolvidos [10.1186/s12992-020-00590-3].\\
Verifica-se então a importância de reduzir os custos associados com o setor, contudo muitas melhorias requerem grande investimento. Desta forma, torna-se atraente a implementação de melhorias ao nível de gestão, requerendo menor investimento ao mesmo tempo que se melhora a eficiência.\\


- Importância do setor de saúde, número de trabalhadores, influencia no PIB/investimento.\\
- Desafios neste setor, falta de trabalhadores, ineficiência geral, falta de investimento.\\
- Alguma comparação com os restantes setores.\\
- Portugal igual a outros países? EU, USA, China, outro?\\

A transição do agendamento manual para o agendamento obtido com ferramentas de otimização permite a utilização mais eficiente da utilização de recursos sem a necessidade de elevado investimento. Ao mesmo tempo, com a construção destas ferramentas, é possível obter novas fontes de informação que permitem, ao decisor, a colocação de novas hipóteses para a melhoria do sistema.\\


\section{Objetivos}
\label{sec:objetivos}

Esta dissertação tem como objetivo o estudo das ferramentas existentes de agendamento, nas suas formas variadas, e a sua implementação num caso real, de forma a verificar a sua utilidade, e validar os resultados que esta apresenta.