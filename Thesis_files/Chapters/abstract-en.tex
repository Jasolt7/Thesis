%!TEX root = ../template.tex
%%%%%%%%%%%%%%%%%%%%%%%%%%%%%%%%%%%%%%%%%%%%%%%%%%%%%%%%%%%%%%%%%%%%
%% abstract-en.tex
%% NOVA thesis document file
%%
%% Abstract in English([^%]*)
%%%%%%%%%%%%%%%%%%%%%%%%%%%%%%%%%%%%%%%%%%%%%%%%%%%%%%%%%%%%%%%%%%%%

\typeout{NT FILE abstract-en.tex}%

Costs related to the health-care industry have grown, therefor there is the need to slow down this growth. A solution may be through the more efficient utilization of resources. A less explored option in the health-care industry is the efficient scheduling of exams throughout the day.\\
Utilizing the nuclear medicine department of the Garcia de Orta Hospital, models were developed with the objective of helping decision-makers answer to two questions: How much time is needed to perform all exams; How many exams can be performed in the allowed time. This problem is a generalization of Job-Shop, and can be said to be a Flexible Multi-Resource Job-Shop with No-Wait problem.\\
This dissertation is separated into three parts, the mapping out of exams and the respective data collection, the formulation of the various models, and the comparison between models and the resolution of practical problem presented.\\
The mapping of the exams was done though interviews with various workers in the department as to capture the needed resources by each activity, following that, the data collection was performed in order to have an estimate of the duration of each activity.\\
For each question there were created four models, a Mixed Integer Linear Programming formulation, as well as three version of Simulated Annealing, for which two different solution codifications were used.\\
The developed models were compared using benchmark instances from existing literature, as well as using practical problems presented by the system in study. However, it was not possible to apply in practice the generated schedules generated by the models.\\
This dissertation allowed the creation of different models that permit the increased efficiency with which exams are shceduled, not only allowing the process to be automated, but also increasing the amount of exams that are done in a given day. 

\keywords{
	Health-care \and
	Optimization \and
	Scheduling \and
	Job-Shop \and
	Process mapping
}
